\documentclass{article}
\usepackage[UTF8]{ctex}
\usepackage{geometry}
\usepackage{natbib}
\usepackage{makecell}
\usepackage{float}
\geometry{left=3.18cm,right=3.18cm,top=2.54cm,bottom=2.54cm}
\usepackage{graphicx}
\pagestyle{plain}	
\usepackage{setspace}
\usepackage{caption2}
\usepackage{datetime} %日期
\renewcommand{\today}{\number\year 年 \number\month 月 \number\day 日}
\renewcommand{\captionlabelfont}{\small}
\renewcommand{\captionfont}{\small}
\begin{document}

\begin{figure}
    \centering
    \includegraphics[width=8cm]{upc.png}

    \label{figupc}
\end{figure}

	\begin{center}
		\quad \\
		\quad \\
		\heiti \fontsize{45}{17} \quad \quad \quad 
		\vskip 1.5cm
		\heiti \zihao{2} 《计算科学导论》课程总结报告
	\end{center}
	\vskip 2.0cm
		
	\begin{quotation}
% 	\begin{center}
		\doublespacing
		
        \zihao{4}\par\setlength\parindent{7em}
		\quad 

		学生姓名:\underline{\qquad  郭志鹏 \qquad \qquad}

		学\hspace{0.61cm} 号:\underline{\qquad 1907010221\qquad}
		
		专业班级:\underline{\qquad 计科1902 \qquad  }
		
        学\hspace{0.61cm} 院:\underline{计算机科学与技术学院}
% 	\end{center}
		\vskip 2cm
		\centering
		\begin{table}[h]
            \centering 
            \zihao{4}
            \begin{tabular}{|c|c|c|c|c|c|c|}
            % 这里的rl 与表格对应可以看到,姓名是r,右对齐的;学号是l,左对齐的;若想居中,使用c关键字。
                \hline
                课程认识 & 问题思 考 & 格式规范  & IT工具  & Latex附加  & 总分 & 评阅教师 \\
                30\% & 30\% & 20\% & 20\% & 10\% &  &  \\
                \hline
                 & & & & & &\\
                & & & & & &\\
                \hline
            \end{tabular}
        \end{table}
		\vskip 2cm
		\today
	\end{quotation}

\thispagestyle{empty}
\newpage
\setcounter{page}{1}
% 在这之前是封面,在这之后是正文
\section{引言}
随着学习的深入,我也完成了这一学期的“计算科学导论”的学习。“计算科学导论”这门课是为我们学习计算机本科专业的新生提供了关于计算机学科的入门介绍。当我们对与学习计算机迷茫、不知所措的时候,“计算科学导论”为我们指明了方向。由于对于网络安全方面的兴趣,所以在这个学期的学习中,我也与小组成员浅显地了解了一下“漏洞扫描”方面的知识。以下是我对“计算科学导论”这门课的体会和对“漏洞扫描”的认识。

\section{对计算科学导论这门课程的认识、体会}
总体说明你的整体认识,再举一、二个例子,从某个角度进一步展开讨论,以支持你的认识。\par

\subsection{对计算科学导论的整体认识}
计算科学是一门有相当深度的学科,对于计算机科学系的学生来说,学习计算科学知识,不仅要知其然,更要知其所以然。而且,计算科学学科知识组织结构庞大,大量的知识在结构上呈现出层次结构和半序结构的特点,没有先修课程或前驱课程的支撑,学习后续课程将是非常困难的。这就需要我们掌握坚实的基础知识,多学习,勤思考。\par
\subsection{计算科学的知识组织结构}
每一个学科都有其自身的学科知识组织结构,计算科学也不例外。经过半个多世纪的高速发展,计算科学已经成为一个分支学科众多,知识组织结构庞大,内容丰富,学问很深的一个学科。通过学习这 一章,我认识到了学习计算机专业的方向有非常多,这一张能帮我整理思路,便于我对未来的规划。
\par
\begin{table}[h]
    \centering
    \caption{计算科学的学科内容按照基础理论、基本开发技术、应用以及它们与硬件设备联系的紧密程度分成三个层面}
\begin{tabular}{c|l|}
% 这里的rl 与表格对应可以看到,姓名是r,右对齐的;学号是l,左对齐的;若想居中,使用c关键字。
    \hline
    计算科学应用层 & \makecell[l]{移动计算与全球定位 \quad 计算机自动控制 \quad 计算机辅助制造 \\ 计算机集成制造系统 \quad 机器人 \quad 计算可视化与虚拟现实 \\ 数据与信息检索 \quad 计算机创作 \quad 计算机网络应用软件 \quad 计算科学 \\ 多媒体信息系统 \quad 计算机辅助设计 \quad 信息、管理与决策系统 \\ 自然语言处理 \quad 模式识别与图像处理技术 \quad 计算机图形学 \quad 计算几何 \\ 人工智能与知识工程 \quad 数据表示与存储 \quad 网络与开放系统互联标准 \\ 软件测试技术 \quad 人机工程学( 人机界面 )}  \\
    \hline
    计算科学专业基础层 & \makecell[l]{软件开发方法学:软件工程技术、程序设计方法学\\、软件开发工具和环境、软件开发规范 \\ 编码理论 \quad 密码学 \quad 计算机体系结构 \quad 程序理论 \\ 数据表示理论与数据库系统 \quad 电子计算机系统基础 \quad 计算机接口与通信 \\ 计算机网络与数据通信技术 \quad 自动推理 \quad 故障诊断与器件测试技术 \\ 容错技术 \quad 汇编技术 \quad 操作系统 \quad 高级语言 \quad 程序设计 \quad 数字系统设计 \\ 符号计算与计算机代数 \quad 数据结构技术 \quad 算法设计与分析 \quad 编译与解释技术 } \\ 
    \hline
    计算科学基础层 & \makecell[l]{控制论基础 \quad 数字系统设计基础 \quad 形式语义学 \quad 网论(Petri网理论) \\ 信息论基础 \quad 框图理论 \quad 算法理论 \quad 可计算性( 递归论) \\ 计算复杂性 \quad 程序设计语言理论 \quad 计算模型( 各种抽象机) \\ 模型论与非经典逻辑 \quad 公理集合论 \quad 形式语言与自动机} \\
    \hline
    数学与物理学层 & \makecell[l]{光电子技术基础 \quad 电路基础 \quad 电子线路基础 \quad 数字与模拟电路基础 \\ 数值分析与计算方法 \quad 大学物理学 \quad 泛函数 \\ 函数论基础( 复变函数、λ演算、泛函分析等)  \\ 概率与数理统计 \quad 常微分方程 \quad 偏微分方程 \quad 集合论与图论 \\ 组合数学 \quad 抽象代数 \quad 梳理逻辑基础 \quad 空间解析几何 \\ 数学分析 \quad 布尔代数 \quad 高等代数 \quad 数论}\\
    \hline
\end{tabular}
    \label{table1}
\end{table}
\citep{zhao}
\subsection{布尔代数}
“计算科学导论”这本书上讲到了布尔代数。布尔代数本身是一种代数系统,但更是一种逻辑系统。当人们将电路与布尔代数建立联系之后,从计算模型到技术支持,存储程序式通用电子数字计算机的设计与制造可谓真正建立在数学基础之上,建立在了逻辑与代数的基础之上,特别是数理逻辑的基础之上。刚刚解除了一点点布尔代数基础,这可以为今后学习计算机逻辑代数,数字逻辑,计算机组成原理,二进制运算以及数理逻辑等课程提供一个基础。\citep{zhao}\par

\section{对漏洞扫描的进一步的思考}
漏洞扫描是指基于漏洞数据库,通过扫描等手段对指定的远程或者本地计算机系统的安全脆弱性进行检测,发现可利用漏洞的一种安全检测(渗透攻击)行为。\par
漏洞扫描器包括网络漏扫、主机漏扫、数据库漏扫等不同种类\citep{doupe2012enemy}\par
\begin{itemize}
    \item {\bf 意义}\par
    如果把网络信息安全工作比作一场战争的话,漏洞扫描器就是这场战争中,盘旋在终端设备,网络设备上空的“全球鹰”。\par
    网络安全工作是防守和进攻的博弈,是保证信息安全,工作顺利开展的奠基石。及时和准确地审视自己信息化工作的弱点,审视自己信息平台的漏洞和问题,才能在这场信息安全战争中,处于先机,立于不败之地。只有做到自身的安全,才能立足本职,保证公司业务稳健的运行,这是信息时代开展工作的第一步。\par
    漏洞扫描器,就是保证这场信息战争胜利的开始,它及时准确的察觉到信息平台基础架构的安全,保证业务顺利的开展,保证业务高效迅速的发展,维护公司,企业,国家所有信息资产的安全。\par
    \item {\bf 定义}\par
    漏洞扫描技术是一类重要的网络安全技术。它和防火墙、入侵检测系统互相配合,能够有效提高网络的安全性。通过对网络的扫描,网络管理员能了解网络的安全设置和运行的应用服务,及时发现安全漏洞,客观评估网络风险等级。网络管理员能根据扫描的结果更正网络安全漏洞和系统中的错误设置,在黑客攻击前进行防范。如果说防火墙和网络监视系统是被动的防御手段,那么安全扫描就是一种主动的防范措施,能有效避免黑客攻击行为,做到防患于未然。\par
    \item {\bf 功能}\par
    1. 定期的网络安全自我检测、评估 \\
    配备漏洞扫描系统,网络管理人员可以定期的进行网络安全检测服务,安全检测可帮助客户最大可能的消除安全隐患,尽可能早地发现安全漏洞并进行修补,有效的利用已有系统,优化资源,提高网络的运行效率。\par
    2. 安装新软件、启动新服务后的检查\\
    由于漏洞和安全隐患的形式多种多样,安装新软件和启动新服务都有可能使原来隐藏的漏洞暴露出来,因此进行这些操作之后应该重新扫描系统,才能使安全得到保障。\par
    3. 网络建设和网络改造前后的安全规划评估和成效检验\\
    网络建设者必须建立整体安全规划,以统领全局,高屋建瓴。在可以容忍的风险级别和可以接受的成本之间,取得恰当的平衡,在多种多样的安全产品和技术之间做出取舍。配备网络漏洞扫描/网络评估系统可以让您很方便的进行安全规划评估和成效检验。\\
    网络的安全系统建设方案和建设成效评估
    4. 网络承担重要任务前的安全性测试\\
    网络承担重要任务前应该多采取主动防止出现事故的安全措施,从技术上和管理上加强对网络安全和信息安全的重视,形成立体防护,由被动修补变成主动的防范,最终把出现事故的概率降到最低。配备网络漏洞扫描/网络评估系统可以让您很方便的进行安全性测试。\par
    5.网络安全事故后的分析调查\\
    网络安全事故后可以通过网络漏洞扫描/网络评估系统分析确定网络被攻击的漏洞所在,帮助弥补漏洞,尽可能多得提供资料方便调查攻击的来源。\par
    6.重大网络安全事件前的准备\\
    重大网络安全事件前网络漏洞扫描/网络评估系统能够帮助用户及时的找出网络中存在的隐患和漏洞,帮助用户及时的弥补漏洞。\par
    7.公安、保密部门组织的安全性检查\\
    互联网的安全主要分为网络运行安全和信息安全两部分。网络运行的安全主要包括以ChinaNet、ChinaGBN、CNCnet等10大计算机信息系统的运行安全和其它专网的运行安全;信息安全包括接入Internet的计算机、服务器、工作站等用来进行采集、加工、存储、传输、检索处理的人机系统的安全。网络漏洞扫描/网络评估系统能够积极的配合公安、保密部门组织的安全性检查。\citep{Fonseca2008Testing}\par
    \item {\bf 分类}\par
    {\bf 依据扫描执行方式不同,漏洞扫描产品主要分为两类,还有针对WEB应用、中间件等}\par
    1.针对网络的扫描器\par
    基于网络的扫描器就是通过网络来扫描远程计算机中的漏洞。比如,利用低版本的DNS Bind漏洞,攻击者能够获取root权限,侵入系统或者攻击者能够在远程计算机中执行恶意代码。使用基于网络的漏洞扫描工具,能够检测到这些低版本的DNS Bind是否在运行。一般来说,基于网络的漏洞扫描工具可以看作为一种漏洞信息收集工具,它根据小同漏洞的特性,构造网络数据包,发给网络中的中的一个或多个目标服务器,以判断某个特定的漏洞是否存在。基于网络的漏洞扫描器包含网络映射( NetworkMapping)和端口扫描功能。基于网络的漏洞扫描器一般结合了Namp网络端口扫描功能,常常用来检测目标系统中到底开放了哪些端口,并通过特定系统中提供的相关端口信息,增强了漏洞扫描器的功能。\par
    {\bf 优点}\par
    1.基于网络的漏洞扫描在操作过程中,不需要涉及到目标系统的管理员。基于网络的漏洞扫描器,在检测过程中,不需要再目标系统上安装任何东西。\par
    2.维护简便。当企业的网络发生了变化的时候,只要某个节点,能够扫描网络中的全部目标系统,基于网络的漏洞扫描器不需要进行调整。\par 
    {\bf 不足之处}\par 
    1.基于网络的漏洞扫描不能直接访问目标系统的文件系统,相关的一些漏洞不能预测到。\par 
    2.基于网络的漏洞扫描不能穿过防火墙。\par 
    3.扫描服务器与目标主机之间通讯过程中的加密机制。控制台与扫描服务器之间的通讯数据包是加过密的,但是,扫描服务器与目标主机之间的通讯数据是没有加密的。这样的话,攻击者就可以利用sniffer工具,来监听网络中的数据包,进而得到各目标集中的漏洞信息。\citep{shenyang}\par 
    2.针对主机的扫描器\par
     基于主机的扫描器则是在目标系统上安装了一个代理(Agent)或者是服务(Services),以便能够访问所有的文件与进程,这也使得基于主机的扫描器能够扫描到更多的漏洞。\par 
     {\bf 优点}\par 
     1.扫描的漏洞数量多。\par 
     2.集中化管理。基于主机的漏洞扫描器通常都有个集中的服务器作为扫描服务器。所有扫描的指令,均从服务器进行控制,这一点与基于网络的扫描器类似。服务器下载到最新的代理程序后,再分发给各个代理。这种集中化管理模式,使得基于主机的漏洞扫描器的部署上,能够快速实现。\par 
     3.网络流量负载小。\par 
     4.通讯过程中的加密机制。\par 
     {\bf 不足之处} \par 
     1.价格。基于主机的漏洞扫描工具的价格,通常由一个管理器的许可证价格加上目标系统的数量来决定,当一个企业网络中的目标主机较多是,扫描工具的价格就非常高。通常,只有实力强大的公司和政府部门才有能力购买这种漏洞扫描工具。\par 
     2.基于主机的漏洞扫描工具需要再目标主机上安装一个代理或服务,而从管理员的角度来说,并不希望在重要的机器上安装自己不确定的软件。\par 
     3.随着索要扫描的网络范围的扩大,在部署基于主机的漏洞扫描工具的代理软件的时候,需要与每个目标系统的用户打交道。这必然延长了首次部署的工作周期
     \citep{lujiaying}\par
     \item {\bf 课堂上问题的回答}\par
     问题;近几年漏洞扫描安全发现事件相对与几年前多了还是少了?为什么?\par
     答:少了。原因是windows安全系统提高了。
\end{itemize}


\section{总结}
学习“计算科学导论”这门可使我受益匪浅,课程全面地阐述了计算学科中的科学问题,包括计算机体系结构与组织、程序设计语言、程序设计基础、算法、信息管理、软件工程、操作系统、人机交互、离散数学、社会职业问题等。并通过大量生动的例子,深入浅出地阐明了计算学科中各领域发展的基本规律,揭示了各领域之间的内在联系,有助于我们更好地了解学科中具有共同的、本质特征的内容。\par
课程运用数学的公理化思想,将整个学科的脉络梳理得清晰、透彻,构建了一个系统化、逻辑化的认知模型,将学科中一些看似凌乱、不相关的知识用一条线顺畅地串了起来。计算科学与技术方法论课程系统全面地为我们介绍了计算科学知识领域划分的过程,涵盖的问题,以及学科的本质。使我们从一开始就有了清晰、明确的方向和认识,学习的过程中不再感到困惑、茫然。\par


\section{附录}
\begin{figure}[H]
	\centering
	\includegraphics[scale=0.25]{github}
	\caption{github账号与网址}
	\label{fig:github}
\end{figure}
	\begin{figure}[H]
		\centering
		\includegraphics[scale=0.1]{guanchazhe}
		\caption{观察者}
		\label{fig:guanchazhe}
	\end{figure}
\begin{figure}[H]
	\centering
	\includegraphics[scale=0.1]{xuexiqiangguo}
	\caption{学习强国}
	\label{fig:xuexiqiangguo}
\end{figure}
\begin{figure}[H]
	\centering
	\includegraphics[scale=0.1]{bilibili}
	\caption{bilibili}
	\label{fig:bilibili}
\end{figure}
\begin{figure}[H]
	\centering
	\includegraphics[scale=0.25]{csdn}
	\caption{CSDN账号与网址}
	\label{fig:csdn}
\end{figure}
\begin{figure}[H]
	\centering
	\includegraphics[scale=0.25]{bokeyuan}
	\caption{博客园账号与网址}
	\label{fig:bokeyuan}
\end{figure}
\begin{figure}[H]
	\centering
	\includegraphics[scale=0.25]{xiaomuchong}
	\caption{小木虫账号与网址}
	\label{fig:xiaomuchong}
\end{figure}



\hspace*{\fill} \\
\bibliographystyle{plain}
\bibliography{references}


\end{document}
